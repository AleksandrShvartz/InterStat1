\documentclass[12pt,a4paper]{article}
\usepackage[utf8]{inputenc}
\usepackage[T2A]{fontenc}
\usepackage[russian]{babel}
\usepackage{amsmath,amssymb}
\usepackage{graphicx}
\usepackage{booktabs}
\usepackage[margin=2cm]{geometry}
\usepackage{float}
\usepackage{hyperref}

\begin{document}

\input{title}

\tableofcontents
\newpage

\section{Цель работы}
Получить практические навыки вычисления интервальных описательных статистик (моды, медиан),
работы с коэффициентом Жаккара и применения методов оптимизации для интервальных данных.
Сравнить эффективность различных функционалов на основе интервальных статистик
для оценивания параметров моделей.

\section{Теоретические сведения}

\subsection{Интервальная арифметика}

\textbf{Интервалом} называется замкнутое подмножество вещественной прямой:
\[
  \mathbf{x} = [\underline{x},\; \overline{x}] = \{x \in \mathbb{R} \mid \underline{x} \le x \le \overline{x}\}.
\]
Середина и радиус интервала:
\[
  \operatorname{mid}\mathbf{x} = \frac{\underline{x} + \overline{x}}{2}, \qquad
  \operatorname{rad}\mathbf{x} = \frac{\overline{x} - \underline{x}}{2}.
\]

Основные арифметические операции над интервалами $\mathbf{x}=[\underline{x},\overline{x}]$
и $\mathbf{y}=[\underline{y},\overline{y}]$:
\begin{align}
  \mathbf{x} + \mathbf{y} &= [\underline{x}+\underline{y},\; \overline{x}+\overline{y}], \\
  a + \mathbf{x} &= [a + \underline{x},\; a + \overline{x}], \quad a \in \mathbb{R}, \\
  t \cdot \mathbf{x} &= \begin{cases}
    [t\,\underline{x},\; t\,\overline{x}], & t \ge 0, \\
    [t\,\overline{x},\; t\,\underline{x}], & t < 0.
  \end{cases}
\end{align}

Пересечение двух интервалов:
\[
  \mathbf{x} \cap \mathbf{y} = [\max(\underline{x}, \underline{y}),\; \min(\overline{x}, \overline{y})],
\]
определено при $\max(\underline{x},\underline{y}) \le \min(\overline{x},\overline{y})$.

\subsection{Коэффициент Жаккара}

Коэффициент Жаккара (Jaccard index) для двух интервалов определяется как отношение
длины пересечения к длине объединения:
\[
  Ji(\mathbf{x}, \mathbf{y}) = \frac{\operatorname{wid}(\mathbf{x} \cap \mathbf{y})}{\operatorname{wid}(\mathbf{x} \cup \mathbf{y})},
\]
где $\operatorname{wid}\mathbf{x} = \overline{x} - \underline{x}$ --- ширина интервала,
а объединение берётся как интервальная оболочка:
$\mathbf{x} \cup \mathbf{y} = [\min(\underline{x},\underline{y}),\; \max(\overline{x},\overline{y})]$.

Для набора пар интервалов $\{(\mathbf{x}_i, \mathbf{y}_i)\}_{i=1}^n$ коэффициент Жаккара
обобщается как:
\[
  Ji = \frac{\displaystyle\sum_{i=1}^{n} \operatorname{wid}(\mathbf{x}_i \cap \mathbf{y}_i)}
            {\displaystyle\sum_{i=1}^{n} \operatorname{wid}(\mathbf{x}_i \cup \mathbf{y}_i)}.
\]

\subsection{Интервальная мода}

Интервальная мода $\operatorname{mode}\mathbf{X}$ выборки интервалов
$\mathbf{X} = \{\mathbf{x}_1, \ldots, \mathbf{x}_n\}$ определяется как интервал
максимальной глубины, т.е. точка на вещественной прямой, покрытая наибольшим
числом интервалов из выборки. Формально:
\[
  \operatorname{mode}\mathbf{X} = \arg\max_{\mathbf{z}} \operatorname{depth}(\mathbf{z}, \mathbf{X}),
\]
где $\operatorname{depth}(\mathbf{z}, \mathbf{X}) = \#\{i : \mathbf{z} \subseteq \mathbf{x}_i\}$.
На практике мода вычисляется алгоритмом <<заметающей прямой>> (sweep line)
за время $O(n \log n)$.

\subsection{Интервальные медианы}

\textbf{Медиана Крейновича} $\operatorname{med}_K$ определяется через медианы границ:
\[
  \operatorname{med}_K \mathbf{X} = [\operatorname{median}(\underline{x}_1, \ldots, \underline{x}_n),\;
  \operatorname{median}(\overline{x}_1, \ldots, \overline{x}_n)].
\]

\textbf{Медиана Пролубникова} $\operatorname{med}_P$ определяется через медианы
середин и радиусов:
\[
  \operatorname{med}_P \mathbf{X} = [m - r,\; m + r],
\]
где $m = \operatorname{median}(\operatorname{mid}\mathbf{x}_1, \ldots, \operatorname{mid}\mathbf{x}_n)$,
$r = \operatorname{median}(\operatorname{rad}\mathbf{x}_1, \ldots, \operatorname{rad}\mathbf{x}_n)$.

При одинаковых радиусах всех интервалов обе медианы совпадают:
$\operatorname{med}_K = \operatorname{med}_P$.

\subsection{Постановка задачи оптимизации}

Рассматриваются две модели связи выборок $\mathbf{X}$ и $\mathbf{Y}$:
\begin{itemize}
  \item \textbf{Аддитивная модель:} $a + \mathbf{X} = \mathbf{Y}$, т.е. $a + \mathbf{x}_i = \mathbf{y}_i$ для всех $i$.
  \item \textbf{Мультипликативная модель:} $t \cdot \mathbf{X} = \mathbf{Y}$, т.е. $t \cdot \mathbf{x}_i = \mathbf{y}_i$ для всех $i$.
\end{itemize}

Для аддитивной модели функционал вычисляется как:
\[
  F(a) = Ji\bigl(\{a + \mathbf{x}_i\},\; \{\mathbf{y}_i\}\bigr)
       = \frac{\sum_{i=1}^{n} \operatorname{wid}\bigl((a+\mathbf{x}_i) \cap \mathbf{y}_i\bigr)}
              {\sum_{i=1}^{n} \operatorname{wid}\bigl((a+\mathbf{x}_i) \cup \mathbf{y}_i\bigr)},
\]
аналогично для мультипликативной модели $F(t) = Ji(\{t \cdot \mathbf{x}_i\},\, \{\mathbf{y}_i\})$.

Точечная оценка параметра $s \in \{a, t\}$ находится максимизацией функционала:
\[
  \hat{s} = \arg\max_s F(s, \mathbf{X}, \mathbf{Y}),
\]
где рассматриваются четыре варианта функционала $F$:
\begin{enumerate}
  \item[\textbf{B.1}] $F(s) = Ji(s, \mathbf{X}, \mathbf{Y})$ --- по полным выборкам,
  \item[\textbf{B.2}] $F(s) = Ji(s, \operatorname{mode}\mathbf{X}, \operatorname{mode}\mathbf{Y})$ --- по модам,
  \item[\textbf{B.3}] $F(s) = Ji(s, \operatorname{med}_K\mathbf{X}, \operatorname{med}_K\mathbf{Y})$ --- по медианам Крейновича,
  \item[\textbf{B.4}] $F(s) = Ji(s, \operatorname{med}_P\mathbf{X}, \operatorname{med}_P\mathbf{Y})$ --- по медианам Пролубникова.
\end{enumerate}

Интервальная оценка параметра строится как множество значений $s$, для которых
$F(s) \ge \alpha \cdot F(\hat{s})$, где $\alpha \in (0, 1)$ --- уровень доверия (в данной работе $\alpha = 0.95$).

\section{Исходные данные}

Загружены два файла данных диагностики томсоновского рассеяния:
\begin{itemize}
  \item \texttt{-0.205\_lvl\_side\_a\_fast\_data.bin} --- выборка $\mathbf{X}$,
  \item \texttt{0.225\_lvl\_side\_a\_fast\_data.bin} --- выборка $\mathbf{Y}$.
\end{itemize}

Разрядность АЦП $N = 14$ бит, откуда $2^N = 16384$.
Формула перевода кодов АЦП в вольты: $V = \text{Code}/2^N - 0.5$.

Каждое измерение представляется интервалом $\mathbf{x}_i = [V_i - r,\; V_i + r]$ с радиусом
$r = \operatorname{rad}\mathbf{x} = \operatorname{rad}\mathbf{y} = \frac{1}{2^{14}} = 0.0000610352$.

\begin{itemize}
  \item $|\mathbf{X}| = 819200$ интервалов
  \item $|\mathbf{Y}| = 819200$ интервалов
\end{itemize}

\section{Интервальные статистики}

\begin{table}[H]
\centering
\caption{Интервальные статистики выборок}
\begin{tabular}{lcc}
\toprule
Статистика & $\mathbf{X}$ & $\mathbf{Y}$ \\
\midrule
$\text{mode}$ & $[-0.171448,\; -0.171387]$ & $[0.172058,\; 0.172119]$ \\
$\text{med}_K$ & $[-0.170593,\; -0.170471]$ & $[0.172913,\; 0.173035]$ \\
$\text{med}_P$ & $[-0.170593,\; -0.170471]$ & $[0.172913,\; 0.173035]$ \\
\bottomrule
\end{tabular}
\end{table}

\section{Оптимизация параметров}

Оптимизация проводилась методом перебора по равномерной сетке в два этапа:
(1) грубый поиск на 500--1000 точках в широком диапазоне,
(2) уточнение на 500 точках в окрестности найденного оптимума.
Интервальная оценка параметра строилась как множество значений $s$,
для которых $F(s) \ge 0.95 \cdot Ji_{\max}$.

\subsection{Аддитивная модель: $a + \mathbf{X} = \mathbf{Y}$}

\begin{table}[H]
\centering
\caption{Результаты оптимизации для аддитивной модели}
\begin{tabular}{lccc}
\toprule
Функционал & $\hat{a}$ & $Ji_{\max}$ & Интервальная оценка $a$ (95\%) \\
\midrule
B.1 & 0.342365 & 0.000061 & $[0.341210,\; 0.344737]$ \\
B.2 & 0.343506 & 0.993292 & $[0.343504,\; 0.343507]$ \\
B.3 & 0.343506 & 0.996640 & $[0.343503,\; 0.343508]$ \\
B.4 & 0.343506 & 0.996640 & $[0.343503,\; 0.343508]$ \\
\bottomrule
\end{tabular}
\end{table}

\subsection{Мультипликативная модель: $t \cdot \mathbf{X} = \mathbf{Y}$}

\begin{table}[H]
\centering
\caption{Результаты оптимизации для мультипликативной модели}
\begin{tabular}{lccc}
\toprule
Функционал & $\hat{t}$ & $Ji_{\max}$ & Интервальная оценка $t$ (95\%) \\
\midrule
B.1 & -1.014830 & 0.000052 & $[-1.031182,\; -1.006814]$ \\
B.2 & -1.003915 & 0.988506 & $[-1.003923,\; -1.003907]$ \\
B.3 & -1.014314 & 0.985888 & $[-1.014339,\; -1.014298]$ \\
B.4 & -1.014314 & 0.985888 & $[-1.014339,\; -1.014298]$ \\
\bottomrule
\end{tabular}
\end{table}

\section{Графики}

\begin{figure}[H]
\centering
\includegraphics[width=\textwidth]{results.png}
\caption{Зависимость функционала $F(s) = Ji$ от параметра для аддитивной (верхний ряд)
и мультипликативной (нижний ряд) моделей.
Красная штриховая --- $s_{\max}$, зелёная пунктирная --- значение $Ji_{\max}$.}
\end{figure}

\section{Сравнение результатов и выводы}

\begin{enumerate}
  \item Все четыре функционала дают согласованные оценки параметра $a$: среднее $\hat{a} \approx 0.3432$, разброс $\Delta a = 0.0011$.
  \item Для мультипликативной модели $\hat{t} \approx -1.0118$. Коэффициент Жаккара для аддитивной модели выше во всех четырёх функционалах, что указывает на лучшее соответствие аддитивной модели данным.
  \item При работе с полными выборками (B.1) $Ji_{\max} \approx 0.000$, тогда как для
    статистик (B.2--B.4) $Ji_{\max} \approx 0.99$.
    Малое значение для B.1 объясняется тем, что при 819200 парах интервалов с малым радиусом
    ($r \approx 6{,}1 \cdot 10^{-5}$) и значительным разбросом середин большинство пар
    $(a + \mathbf{x}_i) \cap \mathbf{y}_i$ имеют малое перекрытие.
    Для статистик же оптимизация проводится по одной паре представительных интервалов,
    что даёт почти полное совпадение при оптимальном $s$.
  \item Использование интервальных статистик (мода, медианы Крейновича и Пролубникова)
    вместо полных выборок значительно ускоряет вычисления при близких точечных оценках параметров.
  \item Медиана Крейновича и медиана Пролубникова дают идентичные результаты,
    поскольку все радиусы интервалов одинаковы ($\operatorname{rad}\mathbf{x}_i = \text{const}$).
  \item Для аддитивной модели медианы (B.3, B.4) дают $Ji_{\max} = 0.9966$, что выше моды $Ji_{\max} = 0.9933$.
    Для мультипликативной модели, напротив, мода ($Ji_{\max} = 0.9885$) превосходит медианы ($Ji_{\max} = 0.9859$).
    Это показывает, что выбор представительной статистики зависит от модели.
  \item Интервальные оценки по полным выборкам (B.1) на 3 порядка шире, чем по статистикам (B.2--B.4):
    $\Delta a_{\text{B.1}} \approx 0.004$ против $\Delta a_{\text{B.2}} \approx 2 \cdot 10^{-6}$.
    Это следствие агрегации данных в одну пару представительных интервалов.
\end{enumerate}

\section*{Репозиторий}
Исходный код доступен по ссылке: \url{https://github.com/AleksandrShvartz/InterStat1}

\end{document}
